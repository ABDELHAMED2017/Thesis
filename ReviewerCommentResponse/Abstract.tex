\documentclass{article}
\usepackage{subfigure}
\usepackage{todonotes}
\usepackage{times,amsmath,epsfig}
%\title{Thesis\\ Responses to Reviewers' Comments}
\begin{document}
%\maketitle


\begin{table}[h]
%	\centering
%	\caption{Popular window-based FIR filter lengths}
%	\label{tab:lengthFIR}
%	\renewcommand{\arraystretch}{1.5}
	\begin{tabular}{ll}


{\large Thesis Title }&: {\Large Techniques for Multi-Standard Cognitive Radios on FPGAs}\\
&\\
{\large Name of Student} &: {\Large Pham Hung Thinh} \\
&\\
{\large Supervisor  }& : {\Large Asst Prof Suhaib A Fahmy}\\
& \\
& 
	\end{tabular}
\end{table}

\section*{Abstract}
\large{
The thesis explores techniques for enabling cognitive radio design on field programmable gate arrays (FPGAs). 
We demonstrate the strengths of FPGAs in offering a high throughput, low-power baseband platform, and develop a flexible Orthogonal Frequency Division Multiplexing (OFDM) baseband chain with high-level control and support for multiple standards. 
We present contributions in OFDM synchronisation to enable more robust radios in harsher channels, and tolerating less precise RF components. We also present a novel technique for managing out of band leakage to enable more efficient spectral use in a dynamic spectrum allocation setting. 
For each of these approaches, we design, optimise, and characterise working hardware implementations of the required modules, with a focus on 
flexibility and low power.
Finally, we present an approach for applying FPGA partial reconfiguration to minimise reconfiguration time when a radio switches modes, allowing intermediate data to be buffered and processed after reconfiguration is complete. 
These contributions form an important foundation in building a fully functional prototyping platform for cognitive radio systems.
}

\end{document}
