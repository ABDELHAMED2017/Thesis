\documentclass{article}
\usepackage{subfigure}
\usepackage{todonotes}
\usepackage{times,amsmath,epsfig}
%\title{Thesis\\ Responses to Reviewers' Comments}
\begin{document}
%\maketitle

\section*{Reply to Examiner No. 3}
\begin{table}[h]
	%	\centering
	%	\caption{Popular window-based FIR filter lengths}
	%	\label{tab:lengthFIR}
	%	\renewcommand{\arraystretch}{1.5}
	\begin{tabular}{ll}
		{\large From Candidate} &: {\large Pham Hung Thinh} \\
		{\large Degree }& : {\large Doctor of Philosophy (PhD)}\\
		{\large Thesis Title }&: {\large Techniques for Multi-Standard Cognitive Radios on FPGAs}\\
		& \\
		& 
	\end{tabular}
\end{table}
\begin{quote}
\emph{Given the tremendous growth in wireless communications, this thesis addresses a relevant and a highly topical problem. It provides a very comprehensive treatment of different aspects of cognitive radios and FPGA-based reconfigurable computing. It brings these two topics together and provides convincing evidence of FPGAs being highly suitable candidates for cognitive radio platforms. Towards this, it solves a number of technical problems that were obstacles in the path of such FPGA-based implementation.}
	
\emph{The work presented in this thesis has been published in a number of top-tier journals and conferences with highly competitive acceptance rates. These include the IEEE transactions on Very Large Scale Integration Systems (TVLSI) and the International Conference on Field Programmable Logic and Applications (FPL). This also points to the relevance of this work and its high technical standards.}

\emph{The thesis has been well written, it provides an excellent survey of the current state-of-the-art, as well as the technical background that is required to understand this work. It is also well organized and provides a very good motivation for this work, in addition to systematically introducing the different technical challenges involved and structuring the proposed solutions to address them. The contributions of the thesis not only include novel techniques and algorithms, but also their implementations on real FPGA platforms along with their evaluations.}
\end{quote}

I would like to thank the examiner for reading this thesis so carefully and providing these constructive review comments.

\begin{table}[h]
	\begin{tabular}{ll}
		& \\
		& \\
		& \\
		& \\
		& \\
		& \\
		
		Signature and Date 
	\end{tabular}
\end{table}


\end{document}
