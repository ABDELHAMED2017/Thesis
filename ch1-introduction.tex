%!TEX root = main_thesis.tex
%---------------------------------------------------------------------------------
\chapter{Introduction}
\label{chap:introduction}
%---------------------------------------------------------------------------------

Wireless transmission plays a key role in our everyday lives, and has enabled the exponential growth in connectivity that we have witnessed over the last few decades.
An exponential increase in the number of users and nodes, and the throughput demanded between them, means significant developments are crucial in the fundamental methods by which this wireless communication is enabled.
The previous approach of defining fixed wireless standards for use in fixed portions of radio spectrum is giving way to a more dynamic approach to exploiting this scarce resource.
Practical studies have shown that many licensed bands are relatively unused across time and frequency~\cite{FCC2002}.
To improve the efficiency of radio spectrum use, the concept of unlicensed users temporarily reusing unused spectrum in licensed bands is currently being researched.
This concept is known as dynamic spectrum access (DSA)~\cite{Minden}.
Wireless communication systems for realising DSA must be reconfigurable, to support different radio standards in different environments, and adaptive, to react to changing channel conditions without interfering with licensed users and other unlicensed opportunistic users.

A cognitive radio (CR) is a node that is able to adapt its parameters to optimise performance based on interaction with the environment, as well as to perform DSA.
A cognitive radio can modify parameters such as transmit power, coding rate, frame size, bandwidth, and centre frequency, in real time, to obtain suitable performance in a changing, environment.
The radio baseband should also be reconfigurable to enable support for multiples standards such as WiFi, WiMAX, GSM, WCDMA, and other defined access schemes.
More advanced adaptive standards are intrinsically flexible and this flexibility should be managed to optimise performance in the channel.

This thesis explores techniques for enabling the mapping and implementation of dynamic cognitive radios on FPGA platforms.
By dynamic, we are referring to the ability to modify baseband processing to suit difference transmission standards and scenarios.
The cognitive portion of a radio is where decisions are made to modify properties of the communication.
In the context of our work, we are interested in providing a generalised interface for cognitive radio designers to be able to leverage the dynamic capabilities of the hardware platform at a higher level.

FPGAs are silicon devices that allow us to build customised hardware datapaths for a variety of applications.
By exploiting parellelism inherent in many algorithms, it is possible to develop implementations that are significantly faster than equivalent software running on general purpose processors.
FPGAs have long been established as a platform of choice in signal processing due to their suitability for parallel bit-level architectures that align well with many signal processing algorithms~\cite{cummings1999}.
Another key capability of FPGAs that makes them attractive for cognitive radios is their reconfigurability.
The hardware implemented on an FPGA can be modified at runtime, thereby enabling the dynamic capability required for implementing cognitive radios.
While a number of radio research groups have used FPGAs in their platforms, we believe our FPGA expertise can help develop a platform that is both easier to use, and that exploits the more advanced capabilities of FPGAs.

% %---------------------------------------------------------------------------------
 \section{Background}
% %---------------------------------------------------------------------------------

 Orthogonal Frequency Division Multiplexing (OFDM) has often been adopted for implementation within the CR field, and is a prime candidate for DSA networks in which the radio is required to be spectrally aware and able to dynamically access idle parts of the spectrum.
 OFDM is an efficient multi-carrier modulation technique that provides robustness to frequency selective channels and the DSA ability based upon spectrum pooling, where unlicensed users may temporarily access spectral resources during the idle periods of licensed users~\cite{JondralMarch2004}.
 When utilising a given bandwidth of spectrum, OFDM sub-carriers that cause interference with licensed users can be selectively disabled.
 This technique is referred to as non-contiguous orthogonal frequency multiplexing (NC-OFDM)~\cite{MindenJune2006}.

 The lower priority of CRs raises a challenge in term of transmission capability and quality of service. When the spectrum allowed for a CR system is fully occupied by PUs and IUs, the transmission of CRs can be blocked.
 Multiple Standard Cognitive Radios (MSCRs) are able to operate in multiple frequency bands with different specified standards.
 MSCRs are hence a more flexible generalisation of CRs as they can operate across different bands and standards to increase transmission capability and enhance bandwidth efficiency.

%---------------------------------------------------------------------------------
\section{Motivation and Objectives}
%---------------------------------------------------------------------------------

MSCR requires a platform that provides sufficient flexibility, high computational throughput, and ideally power efficiency.
Most practical CRs are built using powerful general purpose processors to achieve flexibility through software, however they fail to achieve the computational throughput and also tend to suffer from high power consumption.
Multiple processor-on-chip (MPoC) architectures can enhance the throughput;
however, these platforms require problems to be formulated in a way suitable for parallel programming, and tend to need additional memories to buffer data transferred between parallel processes.
Moving data between these processes can consume significant power in practice.
Conversely, custom hardware designs such as application specific integrated circuits (ASICs) offer highly efficient computation, high throughput and possibly low power dissipation; however, they suffer from a lack of flexibility, in which operating parameters tend to need to be fixed at design time.
In an application area with fast moving standards and occupied by multiple standards, ASICs are likely to be inefficient in terms of both time and cost for MSCR.
However, modern FPGAs which support partial reconfiguration (PR) are an attractive candidate for cognitive radios.
They not only achieve the high performance of a custom data-path implementation, but also offer design flexibility and potentially low power dissipation.

The feasibility of MSCR implementation depends on the ability to flexibly switch system parameters from those specified for one standard to those specified for another.
As mentioned, MCM-based techniques have been widely investigated for next-generation wireless standards.
These techniques divide communication channels into multi-sub-channels that allow a system to perform parallel data transmission over smaller sub-channels.
This allows the techniques to effectively combat amplitude and phase distortion, impulse noise, multipath propagation and so on.
OFDM and Filter Bank Multi Carrier (FBMC) methods are two examples  of multi-carrier modulations.
OFDM modulation has been the dominant technique adopted for multiple applications in high bit-rate wireless communication systems such as Wireless Local Area Networks (WLAN) standardized in IEEE 802.11 and Metropolitan Area Networks (MAN) in IEEE 802.16.
It is effective to perform spectral sensing and carrier allocation for CRs.
Furthermore, OFDM modulation requires a simple and low cost implementation and effectively parameterizes the system to flexibly switch from one standard to another in comparison to FBMC modulation.
OFDM should be a suitable candidate for a MSCR system.
The advantages of coupling OFDM modulation and the FPGA platform are investigated for the feasibility of implementing the proposed low cost, low power MSCR system.
The ability to perform PR in FPGA and allow effective parameterization in OFDM modulation can enable a MSCR to be built for not only current OFDM-based standards such as 802.11, 802.16, and 802.22, but also be potentially able to accept soft upgrades for future OFDM-based standards.

OFDM-based systems have two main intrinsic disadvantages related to synchronization and spectral leakage. They become critical challenges in scenarios of OFDM-based MSCR.
The state of the art synchronization of OFDM-based system can tolerate a small carrier frequency offset (CFO) that leads to highly strict constraints for RF front-end design.
In MSCR systems, an RF front-end is required to have the ability to switch frequency carriers across a wide frequency range according to spectral operating areas.  This tends to suggest that the constraints of small CFO may not be feasible.
Therefore, new synchronization methods that are robust to large CFO should be researched for OFDM-based MSCR systems.
Another challenge is that CRs normally demand small spectral leakage for both in-band and out-of-band of transmitted signals, whereas OFDM is well known to induce significant amounts of spectral leakage.
Pulse shaping techniques have therefore been widely studied to limit the spectral leakage caused by the OFDM signal.
But pulse shaping techniques cannot help to effectively filter out-of-band spectrum in case of small frequency guard because of the present of image spectrum caused by interpolation or DAC operation.
Fortunately, MSCR systems provide flexible parameters that allow the possibility of a frequency guard extending technique which, when combined with an effective pulse shaping technique, can meet the spectral leakage requirements.

The motivation for this research is to study a low power architecture for MSCR based on coupling PR modules and parameterised modules to minimise the adaptation time.
In addition, a novel method for synchronization and an effective pulse shaping technique have been proposed and shown to be suitable for the requirements of MSCR systems.
%---------------------------------------------------------------------------------
\section{Research Contributions}
%---------------------------------------------------------------------------------

This thesis addresses the design of flexible multi-standard cognitive radios on FPGAs. This includes contributions in platforms, and signal processing techniques with the aim of creating a general OFDM-based platform with high level software programmability for adaptation. The contributions made include:
\begin{enumerate}
\item A multipilerless cross-correlation technique for OFDM timing synchronisation demonstrating low power and low resource utilisation on modern FPGA devices.
We show that DSP blocks, while functionally suited to such tasks, increase power consumption and do not offer improved synchronisation accuracy.
We show that wordlength can be optimised to maintain synchronisation performance while minimising area, and also making the design suitable for FPGA devices with a small number of DSP blocks like the low power Xilinx Spartan 6.

\item A novel OFDM synchronisation method combining robust performance with computational efficiency.
We introduce an improved timing metric for synchronisation resulting in significant efficiency improvements over other methods in the literature.
This provides robust fractional CFO estimation and STO estimation in a range of channels.
In particular, it is robust to larger CFO ranges than many state-of-the-art synchronization implementations can handle.
We demonstrate efficient resource usage and reduced power consumption compared to existing methods and this is explored as a fine-grained trade-off between performance and power consumption.

\item A novel IFO estimation technique that reduces both the power and computational cost in robust OFDM implementations.
Performing the IFO cross-correlation using four-fold resource sharing reduces the estimation cost. Meanwhile, adopting a multiplierless technique and carefully optimising word lengths yields significant power reduction, while maintaining sufficient accuracy to meet synchronisation requirements.
Performance is significantly better than conventional techniques, while being much more efficient.
Robust OFDM synchronisation with IFO estimation at baseband is important to allow the RF front-end specification to be relaxed, thus reducing system cost.
In fact, for some multi-standard radios, and applications suffering significant Doppler shift, RF constraints may be infeasible without techniques such as IFO estimation.

\item A novel filtering scheme for adaptively shaping spectral leakage of OFDM signals according to the transmitted power and Spectrum Emission Mask (SEM) requirements.
Recent OFDM-based standards such as 802.11p for vehicular communication and 802.11af for reusing Television White Spaces (TVWS) demand the strict requirements on spectral leakage that raises a tough challenge for radio frequency (RF) front-end circuits.
Our proposed method can achieve the specification for the most stringent SEM of 802.11p.
For 802.11af, it not only meets the requirement for strict SEM filtering but also feasibly increases the spectrum usage by an additional 10 sub-carriers in a basic channel band, compared to conventional techniques, without violating the SEM specifications.

\item A system-level approach for designing multi-standard radios on FPGAs.
We show that by mixing partial reconfiguration (PR) with static parameterised modules, it is possible to minimise reconfiguration time compared to a full PR implementation.
We also show that it is possible to buffer internal data while reconfiguring the radio leading to no loss during reconfiguration, with the proposed mixed technique requiring less storage.
\end{enumerate}

%---------------------------------------------------------------------------------
\section{Organisation}
%---------------------------------------------------------------------------------

This thesis is organized as follows:
Chapter~\ref{chap:BackgroundLiterature} presents comprehensive background on cognitive radio platforms and OFDM techniques. We also present some background knowledge on FPGAs and power considerations. Chapter~\ref{chap:multiplierlesscorrelator} presents the multiplierless correlation technique for OFDM synchronisation. Chapter~\ref{chap:Synchronisation} our combined CFO and STO synchronisation design. Chapter~\ref{chap:SpectralLeakage} presents our work on shaping spectral leakage for cognitive radios, with results for both 802.11p and 802.11af.
Chapter~\ref{chap:MSCR} discusses our system-level design approach for cognitive radio implementation on FPGAs.
Finally, Chapter~\ref{chap:conclusion} concludes the thesis and presents ideas for future work to follow on from that in this thesis.

%---------------------------------------------------------------------------------
\section{Publications}
%---------------------------------------------------------------------------------
Some of the work presented in the thesis has been written up in a number of published and submitted papers listed below:

\begin{enumerate}

\item  T. H. Pham, S. A. Fahmy, and I. V. McLoughlin, ``Low-power correlation for IEEE 802.16 synchronisation on FPGA,'' in \textit{IEEE Transactions on Very Large Scale Integration (VLSI) Systems}, vol. 21, no. 8, pp. 1549 - 1553, Aug. 2013.

\item T. H. Pham, I. V. McLoughlin, and S. A. Fahmy, ``Robust and Efficient OFDM Synchronisation for FPGA-Based Radios,'' in \textit{Circuits, Systems, and Signal Processing}, vol. 33, no. 8, pp. 2475 - 2493, Aug. 2014, Springer.

\item  T. H. Pham, I. V. McLoughlin, and S. A. Fahmy, ``Shaping Spectral Leakage for IEEE 802.11p Vehicular Communications,'' in \textit{Proceedings of IEEE Vehicular Technology Conference (VTC Spring)}, Seoul, Korea, May 2014.

\item T. H. Pham, S. A. Fahmy, and I. V. McLoughlin, ``Efficient Multi-Standard Cognitive Radios on FPGAs,'' PhD Forum Poster in \textit{Proceedings of the International Conference on Field Programmable Logic and Applications (FPL)}, Munich, Germany, September 2014.

\item T. H. Pham, S. A. Fahmy, and I. V. McLoughlin, ``Efficient Integer Frequency Offset Estimation Architecture for Enhanced OFDM Synchronization,'' under review for \textit{IEEE Transactions on Very Large Scale Integration (VLSI) Systems}.

\item T. H. Pham, S. A. Fahmy, and I. V. McLoughlin, ``Spectrally Efficient Emission Mask Shaping for OFDM Cognitive Radios,'' under review for \textit{IEEE Transactions on Communications (TCOM)}.

\item T. H. Pham, S. A. Fahmy, and I. V. McLoughlin, ``Efficient OFDM-based baseband processing for Multi-Standard Cognitive Radios on FPGAs,'' in preparation for submission to \emph{ACM Transactions on Embedded Computing Systems}.

\end{enumerate}
