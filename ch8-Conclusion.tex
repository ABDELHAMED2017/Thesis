%!TEX root = main_thesis.tex
%---------------------------------------------------------------------------------
\chapter{Conclusion and Future Work}
\label{chap:conclusion}
%---------------------------------------------------------------------------------
This thesis has presented techniques for implementing robust, low-power cognitive radios using modern FPGAs. We have shown that the computational performance of FPGAs allows us to develop techniques for more advanced synchronisation, leading to robustness in fading channels, that we can control spectral leakage to satisfy even very strict constraints defined by recent standards, and that a multi-standard radio can be built to minimise reconfiguration time by mixing the use of partial reconfiguration and parameterised modules. The contributions made serve as important building blocks in developing a functional dynamically reconfigurable OFDM cognitive radio.

This chapter concludes the thesis, highlighting the contributions and suggesting ideas for future research.

\section{Summary of Contributions}

This thesis has worked towards the implementation of dynamically configurable cognitive radios on FPGAs.
With this goal in mind, we identified key challenges including robust synchronisation and shaping spectral leakage.
We proposed novel contributions in the field to address these challenges, in each case focusing on applicability in efficient hardware implementations.
We have also explored the system-level challenge of reconfiguration time in radios implemented with partial reconfiguration, and proposed a mixed parametrisation approach for minimising reconfiguration time.

\subsection{Robust, Efficient Synchronisation}

In Chapter~\ref{chap:BackgroundLiterature}, we presented the theory behind OFDM and demonstrated the importance of robust synchronisation, especially in complex channels like those in moving vehicles, or when building flexible radios with less tuned components.
We proposed a multiplierless correlation method for basic timing synchronisation in Chapter~\ref{chap:multiplierlesscorrelator} that can reduce power and area while maintaining performance close to full cross correlation.
We presented a more complex timing metric and its efficient implementation in Chapter~\ref{chap:Synchronisation}, showing robust CFO and STO synchronisation in frequency selective channels, and a computational complexity close to less robust standard methods.
In Chapter~\ref{chap:CFO} we presented a novel method for CFO estimation that tolerates larger CFO variations than existing methods, allowing for flexible radios to be built with support for wide ranging frequency bands without overly expensive RF components.
These methods together offer practical implementations that are more robust than any other published implementations.

\subsection{OFDM Spectrum Shaping}
We discussed in Chapter~\ref{chap:BackgroundLiterature} that OFDM suffers from large sidelobes due to the addition of sub carrier sync pulses.
In the context of cognitive radios opportunistically accessing radio spectrum, typically leads to a reduction in usable bandwidth as some sub-carriers are switched off to reduce potential interference with adjacent channels.
In Chapter~\ref{chap:SpectralLeakage}, we presented a novel method incorporating filtering at the baseband that can offer much better sidelobe suppression, meeting even very stringent requirements in modern CR standards. We demonstrated this approach applied to 802.11p and 802.11af waveforms, and demonstrated its feasibility in hardware.
We also showed that the proposed architecture could flexibly adjust this shaping for moving bands in a CR setting.

\subsection{Multi-Standard Radio Design}

In Chapter~\ref{chap:MSCR}, We presented a detailed analysis of incorporating partial reconfiguration for adapting an FPGA-based CR.
We showed that while PR offered low resource utilisation, long reconfiguration time could be a concern. We showed that mixing PR modules with parametrised modules offered a significant reduction in reconfiguration time, allowing radios to be implemented with internal buffering to overcome data loss during reconfiguration.
We also showed how a clock frequency tuning technique could offer seamless recovery following reconfiguration.

\section{Future Research Directions}

Our achievements in this thesis have helped us identify crucial questions to be explored in future along this line. And while we have been able to test our designs in hardware, building a fully functional radio system is something that still requires some engineering design effort.

\subsection{Efficiently adaptive shaping spectral leakage}
The static strict SEMs in radio systems can guarantee that the systems mitigate the effect of ICI. However, the implementation and power cost may be significant.
To meet a strict SEM, the number of subcarriers may need to be reduced, resulting in decreased the spectrum efficiency and throughput, otherwise the transmission power may be tuned down, leading to a reduction of communication range.
Moreover, the extending frequency guard can maintain the throughput and transmission power but involves increased computational cost.
However, in practice, the adjacent channels are not always occupied and the communication range can change from time to time. Therefore, statically maintaining a very strict SEMs may be redundant.
An interesting research question is whether and how the CR can reduce this redundancy.
The spectral sensing ability of the CR allows the system to recognise the performance of adjacent systems as well as currently allowed communication range.
A novel method is demanded to calculate the dynamic SEM relied upon the performance of adjacent system as well as current allowed communication range.
The dynamic SEM could then be temporarily more relaxed than the static SEM while still guaranteeing it does not causing ICI to adjacent systems.
In addition, the demand of an optimisation approach that takes into account the cost and advantage of the above reduced effort leakage methods can satisfy the dynamic SEM with the smallest cost in terms of throughput, computation and power consumption.

\subsection{Flexible and efficient MSCR platform}
The interface to the higher layer processing is another important factor in building a radio platform for MSCR.
While optimization of low level blocks is equally important, providing a general interface for implementing higher layer processing is important.
This allows radio experts to use the system to investigate cognitive radio techniques without the need for substantial low-level FPGA expertise.
In future work, we aim to build a standardised software interface for this purpose that simplifies the process of retrieving systems status and initiating reconfiguration.
The standardised software interface also allows the MSCR platform to be flexibly and rapidly extended to support new standards.
In addition, in case of increasing the number of standards, pre-determining the next standard in a switching pattern, relies upon the global (or co-operative distributed) spectrum sensing.
Dual PR regions for a PR module could be a good solution to reduce reconfiguration time.
One PR region contains the PR module for the current operating standard while the other region can be reconfigured ready for the standard in the switching pattern.
When frequency bands change, the processing band can therefore instantaneously be switched to the second standard's PR module.
To do so, a new method needs to be studied to in pre-determining the next standard, based on spectrum sensing and this possibly makes use of a predictive system.
The interface between the processing chain and dual PR regions and the switching mechanism also need to be investigated.

\section{Summary}

This thesis has presented key contributions to allow for the design of dynamically reconfigurable cognitive radios on FPGAs. FPGAs, especially recent devices like the Xilinx Zynq, offer an ideal platform for building radios with flexible baseband processing and high level software control. We are confident that these contributions will help make this vision a reality in the near future.
