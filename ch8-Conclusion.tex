%!TEX root = main_thesis.tex
%---------------------------------------------------------------------------------
\chapter{Conclusions and Future Work}
\label{chap:conclusion}
%---------------------------------------------------------------------------------
This thesis has presented techniques for implementing robust, low-power cognitive radios using modern FPGAs. We have shown that the computational performance of FPGAs allows us to develop techniques for more advanced synchronisation, leading to robustness in fading channels, that we can control spectral leakage to satisfy even very strict constraints defined by recent standards, and that a multi-standard radio can be built to minimise reconfiguration time by mixing the use of partial reconfiguration and parameterised modules. The contributions made serve as important building blocks in developing a functional dynamically reconfigurable OFDM cognitive radio.

This chapter concludes the thesis, highlighting the contributions and suggesting ideas for future research.

\section{Summary of Contributions}

This thesis has worked towards the implementation of dynamically configurable cognitive radios on FPGAs.
With this goal in mind, we identified key challenges including robust synchronisation and shaping spectral leakage.
We proposed novel contributions in the field to address these challenges, in each case focusing on applicability in efficient hardware implementations.
We have also explored the system-level challenge of reconfiguration time in radios implemented with partial reconfiguration, and proposed a mixed parametrisation approach for minimising reconfiguration time.

\subsection{Robust, Efficient Synchronisation}

In Chapter~\ref{chap:BackgroundLiterature}, we presented the theory behind OFDM and demonstrated the importance of robust synchronisation, especially in complex channels like those in moving vehicles, or when building flexible radios with less tuned components.
We proposed a multiplierless correlation method for basic timing synchronisation in Chapter~\ref{chap:multiplierlesscorrelator} that can reduce power and area while maintaining performance close to full cross correlation.
We presented a more complex timing metric and its efficient implementation in Chapter~\ref{chap:Synchronisation}, showing robust CFO and STO synchronisation in frequency selective channels, and a computational complexity close to less robust standard methods.
In Chapter~\ref{chap:CFO} we presented a novel method for CFO estimation that tolerates larger CFO variations than existing methods, allowing for flexible radios to be built with support for wide ranging frequency bands without overly expensive RF components.
These methods together offer practical implementations that are more robust than any other published implementations.

\subsection{OFDM Spectrum Shaping}
We discussed in Chapter~\ref{chap:BackgroundLiterature} that OFDM suffers from large sidelobes due to the addition of subcarrier sync pulses.
In the context of cognitive radios opportunistically accessing radio spectrum, typically leads to a reduction in usable bandwidth as some subcarriers are switched off to reduce potential interference with adjacent channels.
In Chapter~\ref{chap:SpectralLeakage}, we presented a novel method incorporating filtering at the baseband that can offer much better sidelobe suppression, meeting even very stringent requirements in modern CR standards. We demonstrated this approach applied to 802.11p and 802.11af waveforms, and demonstrated its feasibility in hardware.
We also showed that the proposed architecture could flexibly adjust this shaping for moving bands in a CR setting.

\subsection{Multi-Standard Radio Design}

In Chapter~\ref{chap:MSCR}, We presented a detailed analysis of incorporating partial reconfiguration for adapting an FPGA-based CR.
We showed that while PR offered low resource utilisation, long reconfiguration time could be a concern. We showed that mixing PR modules with parametrised modules offered a significant reduction in reconfiguration time, allowing radios to be implemented with internal buffering to overcome data loss during reconfiguration.
We also showed how a clock frequency tuning technique could offer seamless recovery following reconfiguration.

\section{Future Research Directions}

Our achievements in this thesis have helped us identify crucial questions to be explored in future along this line. And while we have been able to test our designs in hardware, building a fully functional radio system is something that still requires some engineering design effort.

\subsection{Increasing Spectrum Efficiency with Shaping for NC-OFDM}
The shaping approach presented in Chapter~\ref{chap:SpectralLeakage} offers high out of band attenuation, avoiding interference with neighbouring channels, while allowing use of subcarriers close to adjacent channels.
It could also be applied to in-band shaping in the context of non-contiguous OFDM signals, where some intermediate subcarriers are switched off due to the presence of other uses in intermediate bands.
This could result in significant efficiency improvements in the case of highly fragmented bands, since narrow spaces can still be used effectively

\subsection{Efficiently Adaptive Shaping Spectral Leakage}
The static strict SEMs in radio systems can guarantee that the systems mitigate the effect of ICI. However, the implementation and power cost may be significant.
In practice, adjacent channels are not always occupied and the communication range can change from time to time. Therefore, statically maintaining a very strict SEMs may be wasteful in terms of computation.
One approach would be to dynamically modify the SEM based on occupancy of adjacent channels and current transmit power.
This would mean that in some cases the SEM would be relaxed compared to the static implementation, allowing improved computational efficiency and lower computational power.

\subsection{Standardised Software Interface for Multi-Standard Radio Platform}
We have discussed that there are a variety of different cognitive radio platforms, many of which do include some FPGA capability. However, this interface between higher layers and hardware baseband processing is often implemented in an ad-hoc manner. Our group is exploring how a standardised interface might be proposed to allow higher layer cognitive stacks to be coupled with different baseband systems, including both the data stream interface, and the control. This must entail minimal latency, and enable the strengths of the baseband in terms of high data stream throughput.

\subsection{Alternative MultiCarrier Modulations Techniques}
While OFDM is well established, other multicarrier schemes have been proposed which offer advantages. Demonstrating efficient mapping of these schemes to high performance baseband hardware is important for them to gain traction.
Generalised Frequency Division Multiplexing (GFDM) is a new modulation scheme that overcomes some of OFDM's limitations.
In GFDM, out of band spectrum is compressed by an adjustable pulse shaping filter for individual subcarriers. Transmission efficiency is increased thanks to a two-dimensional data process (grouping data symbols across subcarriers and time slots).
However, it suffers from higher bit error rates (because of the interference between subcarriers and time slots) and requires increased computational resources to implement the prototype filter.
A similar structure means a radio supporting both OFDM and GFDM could be developed, allowing experimentation with this new standard, while offering backwards compatibility.

\subsection{Higher Layer Knowledge to Minimise Reconfiguration Time}
As radios support more standards, switching between them can become time-consuming since many modules need to be reconfigured. One way of overcoming this is using knowledge about the most likely reconfiguration options to prepare the require partial bitstreams in advance, or even to drive the design decision to parameterise some modules that would otherwise use PR.
This information could be forthcoming from higher layers through collaborative decision making or modelling of the dynamic properties of the channel.

\section{Summary}

This thesis has presented key contributions to allow for the design of dynamically reconfigurable cognitive radios on FPGAs. FPGAs, especially recent devices like the Xilinx Zynq, offer an ideal platform for building radios with flexible baseband processing and high level software control. We are confident that these contributions will help make this vision a reality in the near future.
